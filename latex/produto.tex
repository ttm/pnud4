\documentclass[12pt]{article}
\usepackage[usenames,dvipsnames]{color}
\usepackage{listings}
\usepackage{graphicx}
\usepackage{fancyhdr}
\usepackage{framed}
\usepackage[T1]{fontenc}
\usepackage[toc,page]{appendix}
\usepackage[utf8]{inputenc}
\usepackage[brazil]{babel}
\usepackage{fancyvrb}
\usepackage[hmargin=2cm,vmargin=2cm]{geometry}
\usepackage{lastpage}
\usepackage{pdfpages}
\usepackage{makeidx}
\usepackage{hyperref}
\pagestyle{fancy}
\usepackage{enumitem}
% cabecalho e rodapé
\setlength{\headheight}{120pt}
\setlength{\textheight}{550pt}
\renewcommand{\headrulewidth}{0pt}
\lhead{\includegraphics[scale=0.03]{brasao.png}}
%\chead{\includegraphics[scale=0.5]{logo-brasil-sem-pobreza2.png}}
\rhead{\includegraphics[scale=0.5]{logo-pnud.png}}
\cfoot{\textbf{\ProjectCode\ - Inovando a democracia participativa}}
\rfoot{\thepage}

\hyphenation{par-ti-ci-pa-ção}
\bibliographystyle{ieeetr}

% definições sobre o autor e o produto
\newcommand{\MyName}{Renato Fabbri}
\newcommand{\MySurnameForename}{Fabbri, Renato}
\newcommand{\SupervisorName}{Gabriella Vieira Oliveira Gonçalves}
\newcommand{\MyEmail}{renato.fabbri@gmail.com}
\newcommand{\ContractNumber}{2013/000566}
\newcommand{\ContractYear}{2014}
\newcommand{\ProjectCode}{Projeto BRA/12/018}
\newcommand{\NomeSecretaria}{Secretaria-Geral da Presidência da República}
%Q\newcommand{\SiglaSecretaria}{SG/PR}
\newcommand{\SiglaSecretaria}{Secretaria: SNAS }
\newcommand{\ProductNumber}{04}
\newcommand{\ProductTitle}{Proposta de adaptações e incrementos para a interface do portal federal de participação sociais e suas ferramentas}
\newcommand{\ProductSubtitle}{com elementos visuais e de usabilidade para mecanismos de priorização de conteúdos e auto-regulação}
\newcommand{\ProductDescription}{"Proposta de adaptações e incrementos para a interface do portal federal de participação sociais e suas ferramentas, com elementos visuais e de usabilidade para mecanismos de priorização de conteúdos e auto-regulação"}

\newcommand{\ProductValue}{R\$ 21,600 (vinte e um mil e seiscentos reais)}
\newcommand{\ObjetoContratacao}{
Aporte de conhecimentos e tecnologias para especificação de vocabulário e ferramentas assistidas que utilizam processamento de linguagem natural e análise de redes complexas para o conteúdo do portal da participação social.
}
\newcommand{\DataEntrega}{28 de Agosto de 2014}
\newcommand{\PalavrasChave}{reconhecimento de padrões, redes complexas, processamento de linguagem natural, participação social}

% lista de abreviações
\makeindex

\begin{document}

\newgeometry{hmargin=3cm,vmargin=1.5cm}
\begin{center}
\thispagestyle{empty}
{\color{MidnightBlue}

\includegraphics[scale=0.9]{logo-pnud.png}

\vspace{4cm}

{\bf \large \ProjectCode\ - Desenvolvimento de Metodologias
de Articulação e Gestão de Políticas Públicas para Promoção da Democracia
Participativa}

\vspace{1.5cm}

{\bf \large Produto \ProductNumber\ -\ \ProductTitle}

\vspace{1.5cm}

\ProductSubtitle

\vspace{4cm}

\MyName

\vspace{2cm}

}

\includegraphics[scale=0.04]{brasao.png} \\
{\bf \small \NomeSecretaria}

\end{center}
\restoregeometry
\newpage

\newgeometry{hmargin=3cm,vmargin=1.5cm}
\addtolength{\topmargin}{2.5cm}
\thispagestyle{empty}
{\color{MidnightBlue}

{\bf \LARGE Produto \ProductNumber\ -\ \ProductTitle}

\hrulefill

\vspace{1cm}

\begin{center}

{\bf \large Contrato n. \ContractNumber}

\vspace{1.5cm}

{\bf \large Objeto da contratação: \ObjetoContratacao}

\end{center}

\vspace{3.2cm}

Valor do produto: \ProductValue

\vspace{1.2cm}

Data de entrega: \DataEntrega

\vspace{1.2cm}

Nome d@ consultor(a): \MyName

\vspace{1.2cm}

Nome d@ supervisor(a): \SupervisorName

}

\vspace{2cm}

\begin{center}
\includegraphics[scale=0.04]{brasao.png} \\
{\bf \small \NomeSecretaria}
\end{center}

\restoregeometry
\newpage

\newgeometry{hmargin=3cm,vmargin=1.5cm}
\addtolength{\topmargin}{5cm}
\thispagestyle{empty}

\begin{framed}

{\raggedright \MySurnameForename} \\

\ProductTitle: \ProductSubtitle\ / \ContractYear. \\

Total de folhas: \pageref{LastPage} \\

\vspace{1cm}

Supervisor(a): \SupervisorName \\

\SiglaSecretaria \\

\NomeSecretaria \\

Palavras-chave: \PalavrasChave. \\

\end{framed}

\vspace{3cm}

{\raggedright \includegraphics{licenca-cc-by-nc.png} \ Esta obra é licenciada sob
uma licença Creative Commons - Atribuição-NãoComercial. 4.0 Internacional.}

\restoregeometry
\newpage

\tableofcontents
\newpage


\begin{abstract}
Este documento descreve rotinas de priorização de conteúdo e de autorregulação para o portal federal de participação social.\\

{\bf Palavras-chave:} \PalavrasChave.
\end{abstract}
\newpage

\section{Introdução}
\subsection{Contexto e importância da consultoria}
Em confluência com o portal federal de participação social (Participa.br) e o Plano Nacinal de Participação Social (PNPS), esta consultoria propõe métodos de classificação e priorização de conteúdo e formas de autorregulação para o portal. O presente produto apresenta ``formas de priorização de conteúdo e de autorregulação'' implementadas na forma de um sistema de recomendação de recursos para usuários.

\subsection{Contexto e importância do Produto}
\subsubsection{Objetivos}
Este produto tem por objetivo principal a disponibilização de um sistema de recomendação de recursos do participa.br para usuários, tanto via critérios personalizados quanto considerando comunidades e linha editorial. Através deste sistema de recomendação, ficam facilitados, até mesmo prontamente disponíveis, diversos processos de autorregulação, de geração de resumos, de geração de relatórios e análises informativas. Objetivos secundários são:
\begin{itemize}
    \item Disponibilização de uma API HTTP, para uso no participa.br, conforme requisitado pela equipe do Participa.br em diversos itens dos produtos~\cite{prodExtra}.
    \item A exposição destes algoritmos de recomendação aos visitantes em páginas HTML comuns, para edição execução dos trechos de código utilizados pela plataforma federal de participação social. Feito no IPython Notebook levantado para este trabalho~\cite{iNotebook}.
    \item Aproveitamento do endpoint SparQL com os dados do participa.br, fortalecendo as tecnologias de dados linkados e web 3.0.
    \item Em reunião com a consultora Daniela Feitosa, foi delineada a pertinência de um sistema de recomendação de perfis para um ActionIntem em andamento para o participa.br~\cite{AIFeitosa}. Este produto visa suprir esta necessidade através da API disponibilizada.
    \item A entrega das tecnologias livres com simplicidade e boa documentação, favorecendo ao máximo o aproveitamento deste trabalho para melhoras, geração de derivados e novas e independentes tecnologias. Isso pode ser observado no repositório git público deste produto~\cite{pnud4}.
    \item Compatibilizar ao máximo a entrega deste produto às demandas da equipe do participa.br.
    \item Cumprir de forma precisa e pertinente a descrição deste quarto produto no Termo de Referência.
\end{itemize}
\subsubsection{Resultados esperados}
Em um aspecto mais amplo, o resultado do produto é iniciar um processo aberto de apropriação das análises e mecanismos de autorregulação para o portal federal de participação social.

Como resultados diretos destes resultados, constam:
\begin{itemize}
    \item a habilitação para uso da API HTTP para recomendação de recursos do participa.br.
    \item A interface para apreensão e inovação dos algoritmos, não somente exemplificada ou projetada, mas operante e disponível.
    \item Um plano de implementação para o participa.br, que utiliza a API de recomendação para priorização de conteúdo e autorregulação.
    \item Transparência absoluta no trabalho, com toda a documentação e código computacional online em um repositório git que contem o histórico de implementação.
    \item Algoritmos implementados em código de fácil leitura, para facilitar a implementação em Ruby ou Javascript, quando houverem recursos maduros para estas linguagens. (No momento, Python possui mais recursos e mais maduros tanto para redes complexas quanto para processamento de linguagem natural, o que justifica o servidor.)
    \item Habilitação de implementações em andamento, como o plugin de recomendação de perfis~\cite{AIFeitosa}, para recomendar amigos para visitantes ou participantes. 
\end{itemize}
\subsubsection{Caráter inovador}
Centralmente, este trabalho é inovador na aplicação de recursos de análise de redes sociais de forma comunitária, entregando as tecnologias e priorizando a reutilização. Este recurso é de vital interesse para a democracia participativa no contexto atual, com as revoluções da internet e redes sociais. Permite, em última instância, que haja uma inteligência para aproveitamento das estruturas sociais, e que esta inteligência seja pública, transparente, minimizando vetores vigilantistas ou turvos.

Há a inovação na arquitetura em software, apresentando traços de web 3.0 como os dados linkados como base de conhecimento e os múltiplos recursos online acessados no funcionamento usual (ao menos endpoint sparql para acesso aos dados, API HTTP Flask para tratar os dados e gerar estatísticas e estruturas de interesse, participa.br para interface e contexto pertinente).

Há inovação na difusão científica. Dados Linkados, Processamento de Linguagem Natural (PLN) e Redes Complexas (RC) são três termos empregados em textos científicos, consistindo de áreas que recebem pesquisas, revistas dedicadas e até carreiras científicas inteiras. Neste trabalho este conteúdo está em português e se presta a orientar contribuições de outras partes interessadas e de curiosos.

Os métodos de recomendação em si não foram confrontados exaustivamente com a literatura, mas possivelmente possui também inovações nos procedimentos e certamente no contexto de implementação, tanto de relevância social quanto de aparato tecnológico.

\subsubsection{Aparato em hardware}

como este trabalho poderá contribuir suprir uma lacuna de conhecimento e/ou para desenvolver determinada a capacidade institucional da SG/PR

\section{Desenvolvimento}
\subsection{Etapas de desenvolvimento}
\subsection{Justificativa do método}
\subsection{Justificativa das fontes}
\subsection{Confronto entre os resultados esperados e os alcançados}
\section{Usos dos resultados}
\section{Conclusão}
\subsection{Comentários, sugestões, recomendações}
\subsection{Impacto do Produto para a elaboração, gestão e/ou avaliação de políticas públicas de participação social}
\subsection{Como o Produto deverá impactar o público-alvo das políticas públicas a que se refere}
\section{Agradecimentos}
O consultor Renato Fabbri agradece ao Joenio Costa pelo template em \LaTeX para os produtos. Agradece à Daniela Feitosa pela reunião para demanda de recomendação de perfis. Agradece aos supervisores do trabalho realizado em torno do participa.br: Ricardo Poppi e Ronald Costa. Agradece ao labMacambira.sf.net e todas as comunidades de software e cultura livre que compõe esta contribuição.
\newpage
\bibliography{bibliografia}
\newpage
%\listoffigures
\section*{Abreviações e jargão}
\begin{itemize}[label={}]
    \item {\bf RC:              } Redes Complexas
    \item {\bf PLN:             } Processamento de Linguagem Natural
    \item {\bf OPS:             } Ontologia de participação Social
    \item {\bf OPA:             } Ontologia do Participa.br
    \item {\bf MMISSA:          } Monitoramento Massivo e Interativo da Sociedade pela Sociedade para Aproveitamento
    \item {\bf AARS:            } A Análise de Redes Sociais
    \item {\bf MyNSA:           } Monitoring yields Natural Streaming and Analysis
    \item {\bf PNPS:            } Plano Nacional de Participação Social
    \item {\bf RDF:             } Resource Description Framework
    \item {\bf HTTP:            } Hypertext Transfer Protocol
    \item {\bf SPARQL:          } Simple Protocol and RDF Query Language
    \item {\bf endpoint SPARQL: } ponto de acesso, geralmente HTTP, a dados em RDF via buscas em SPARQL.
    \item {\bf Participa.br:    } Portal federal de participação social.
    \item {\bf IPython Notebook:} instância online para rodar scritps Python
    \item {\bf Meteor:          } arcabouço para páginas reativas e com funcionamento distribuído.
    \item {\bf D3js:            } biblioteca de visualização de dados.
\end{itemize}

\newpage
\printindex
\newpage
%\input{listadeanexos.tex}
\appendix
\section{Anexos}
\end{document}
